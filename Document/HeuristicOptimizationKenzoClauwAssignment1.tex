\documentclass[]{article}
\usepackage{amsfonts}
\usepackage{booktabs}
\usepackage{longtable}
\usepackage{float}
\usepackage{color}
\usepackage[dvipsnames]{xcolor}
\newtheorem{hypothesis}{Hypothesis}
\newtheorem{nullhypothesis}{Null Hypothesis}


%opening
\title{Heuristic Optimization : Assignment 1}
\author{Kenzo Clauw}

\begin{document}
	
	\maketitle
	
	\begin{abstract}
		
	\end{abstract}
	
	\section{Run the code}
	\subsection{Compilation}
	Run the make command in the folder of the project.
	./Flowshop is the main script to run the program.
	
	\subsection{Arguments}
	Command line arguments are used to select the different algorithms and their instances.
	These arguments should be executed in the following order :\newline\newline
	\textbf{./flowshop instance improvement neighborhood initial}
	
	
	
	\begin{table}[h!]
		\caption{Syntax of each command line argument.}
		\label{tab:Syntax table}
		\begin{tabular}{ | l | l | p{10cm} |}
			\hline
			Commands & Values & Summary \\ \hline
			
			./flowshop & ./flowshop & Mandatory argument containing the runtime script made by the build command\\ \hline
			
			Instance& instance path & Path of a single instance. \\ 
			& --all & Runs the algorithm on every single instance.\\ \hline
			
			Improvement & --First & iterative first improvement selects the first improving candidate solution \\
			& --Best & iterative best improvement evaluates every solution\\
			\hline
			
			Neighborhood & --Transpose & Swaps each neighbor at position i and i+1 \\
			& --Exchange & Swaps each neighbor at position i and j\\
			& --Insert & Inserts each element at index i through the whole solution space \\
			\hline
			
			Initial & --srz & The initial solution is generated by the srz heuristic\\
			& --Random & the initial solution is generated at random\\
			\hline
		\end{tabular}
	\end{table}
	
	
	
	
	\subsection{Example scenario 1}
	
	\textbf{./flowshop --all}\newline
	
	This scenario will run every single algorithm on each instance.\newline
	The script will then output the runtime of each algorithm per instance in the results.runtime.txt file.\newline
	The score of each algorithm per instance compared to the best solution is contained in the results.score.txt file.
	
	\subsection{Example scenario 2}
	This scenario will run a single algorithm on a specific instance. 
	The purpose of this scenario is to test the individual algorithms on certain instance.
	
	\textbf{./flowshop ./instances/50 --first --transpose --random}\newline
	
	
	
	
	\section{Code Structure}
	This section describes the implementation of assignment 1 written in C++ 11.
	The application is an extension of the code provided by the teacher.\newline
	For each class, the description of each function is as follows :
	\begin{enumerate}  
		\item \textbf{Flowshop}\newline
		Parses the command line arguments to the application and determines each algorithm and instance combination.
		
		\item \textbf{Run}\newline
		Contains the code that is necessary to run the algorithms on each of the instances.
		
		
		\item \textbf{Pfspinstance}\newline
		Compared to the original project there are a few extra methods added.
		When creating the instance there will be an extra 2 dimensional lookup-table used by the Compute\_WCT\_from method.
		The purpose of the Compute\_WCT\_from method is to recompute the value of the WCT starting at a certain job.
		
		
		
		
		\item \textbf{Solution}\newline
		The solution class contains a single solution vector.
		Storage move left/right moves an element in the solution vector from a certain position i to the position j(right) or from position j to i (left).
		Most of these methods are related to navigation, the main reason behind this is to make an abstraction allowing us to move jobs without having to worry about the underlying vector.
		
		
		
		
		
		
		
		
		\item \textbf{Neighborhood}\newline
		Contains all of the algorithms that are necessary to retrieve the neighborhood of the iterative improvement algorithm.
		Each neighborhood algorithm returns a boolean to determine if an improvement can be made.
		Choose\_neighborhood will then function as a strategy method that executes on of the possible neighborhood algorithms and return the improvement boolean.
		This implementation of the neighborhood makes it easy for the variable neighborhood descent algorithm to just run one of the neighborhood algorithms until an improvement is made.
		
		
		
		
		
		
		
		
		.
		
		
		
		
		
	\end{enumerate}
	
	
	
	
	\section{Results}
	
	\subsection{Exercise 1}
	This section contains the average relative percentage of the algorithm scores compared to the best solution score.
	\begin{table}[H]
		\centering
		\caption{Average percentage deviation from the best known solutions}
		\label{tab:table1}
		\begin{tabular}{l{|}lll}
			\toprule
			Algorithm & 50 jobs & 100 jobs & All\\
			\midrule
			First Random Transpose &31.986687&39.415857&35.701272\\
			First Random Exchange & 1.988601&1.681809&1.835205\\
			First Random Insert &2.474552&2.585864&2.530208\\
			First Srz Transpose &8.867471&9.936982&9.402226\\
			First Srz Exchange &2.405070& 2.378449&2.391759\\
			First Srz Insert &1.578144 &2.174206&1.876175\\
			Best Random Transpose &32.5355873&41.166533&36.851060\\
			Best Random Exchange & 3.987419&4.783399& 4.385409\\
			Best Random Insert &3.482139& 4.383328&3.932733 \\
			Best Srz Transpose &8.679424&10.103119&9.391272\\
			Best Srz Exchange &3.707166& 4.255597&3.981381 \\
			Best Srz Insert &2.449823&3.405681&2.927752\\
			\bottomrule
		\end{tabular}
	\end{table}
	
	
	
	\begin{table}[H]
		\centering
		\caption{Average computation time for each number of jobs in seconds}
		\label{tab:table1}
		\begin{tabular}{l{|}lll}
			\toprule
			Algorithm & 50 jobs & 100 jobs & All\\
			\midrule
			First Random Transpose & 0.0018666667 & 0.0266333333 & 0.0142500000 \\ 
			First Random Exchange & 0.4950000000 & 12.3558333333 & 6.4254166667 \\  
			First Random Insert & 0.3066666667 & 6.6004666667 & 3.4535666667 \\ 
			First Srz Transpose & 0.0012000000 & 0.0143000000 & 0.0077500000 \\ 
			First Srz Exchange & 0.1896333333 & 3.8876333333 & 2.0386333333 \\
			First Srz Insert & 0.2947333333 & 6.6719666667 & 3.4833500000 \\ 
			
			Best Random Transpose & 0.0033000000 & 0.0335333333 & 0.0184166667 \\ 
			Best Random Exchange & 0.1510666667 & 2.3577666667 & 1.2544166667 \\ 
			Best Random Insert & 0.2591000000 & 4.1130000000 & 2.1860500000 \\ 
			Best Srz Transpose & 0.0021333333 & 0.0190333333 & 0.0105833333 \\ 
			Best Srz Exchange & 0.1025333333 & 1.5269666667 & 0.8147500000 \\ 
			Best Srz Insert & 0.1660666667 & 2.6129333333 & 1.3895000000 \\ 
			
			
			\bottomrule
		\end{tabular}
	\end{table}
	
	We notice that the algorithms with the transpose neighborhood are the worst algorithms but they take less time to compute.\newline
	The algorithms with the best pivoting rule take the most time to complete, this is because they have to find every candidate solution in the entire neighborhood search space.\newline
	On average, the algorithms with the insert or exchange neighborhood have a better score but need more computational time, this is because these algorithms have quadratic performance compared to the linear performance of the transpose neighborhood.
	First Random Exchange is the best scoring algorithm, but also takes the longest time to complete.
	
	
	
	\subsection{Exercise 2}
	
	
	
	\begin{table}[H]
		\centering
		\caption{Average percentage deviation from the best known solutions}
		\label{tab:table1}
		\begin{tabular}{l{|}lll}
			\toprule
			Algorithm & 50 jobs & 100 jobs & All\\
			\midrule
			First Random Vnd1 &1.690726&1.969149  &1.829937 \\ 
			First Srz Vnd1 &1.982103 &1.982103&2.406998\\ 
			First Random Vnd2 &2.017300&2.915175  &2.466238\\ 
			First Srz Vnd2 & 2.086412&2.796580   &2.441496\\ 
			
			
			
			\bottomrule
		\end{tabular}
	\end{table}
	
	\begin{table}[H]
		\centering
		\caption{Average computation time for each number of jobs in seconds}
		\label{tab:table1}
		\begin{tabular}{l{|}lll}
			\toprule
			Algorithm & 50 jobs & 100 jobs & All\\
			\midrule
			First Random Vnd1 & 0.2193666667 & 3.8318000000 & 2.0255833333 \\ 
			First Srz Vnd1 & 0.1342000000 & 1.8801333333 & 1.0071666667 \\ 
			First Random Vnd2 & 0.1729333333 & 3.1281000000 & 1.6505166667 \\ 
			First Srz Vnd2 & 0.1324333333 & 2.5647666667 & 1.3486000000 \\ 
			
			\bottomrule
		\end{tabular}
	\end{table}
	
	\begin{table}[H]
		\centering
		\caption{Percentage improvement over the usage of a single neighborh
			ood}
		\label{tab:table1}
		\begin{tabular}{l{|}llll}
			\toprule
			Algorithm&Vnd1 Random&Vnd1 Srz&Vnd2 Random&Vnd2 Srz\\
			\midrule
			First Random Transpose &\color{OliveGreen}99\%
			&\color{OliveGreen}99\%
			&\color{OliveGreen}99\%
			&\color{OliveGreen}99\%
			\\ 
			
			First Random Exchange  &\color{Red}-252\%
			&\color{Red}-168\%
			&\color{Red}-163\%
			&\color{Red}-164\%
			\\  
			First Random Insert  &\color{Red}-88\%
			&\color{Red}-43\%
			&\color{Red}-41\%
			&\color{Red}-41\%
			\\
			First Srz Transpose  &\color{OliveGreen}99\%
			&\color{OliveGreen}99\%
			&\color{OliveGreen}99\%
			&\color{OliveGreen}99\%
			\\ 
			First Srz Exchange  &\color{Red}-11\%
			&\color{OliveGreen}15\%
			&\color{OliveGreen}16\%
			&\color{OliveGreen}16\%
			\\
			First Srz Insert  &\color{Red}-90\%
			&\color{Red}-44\%
			&\color{Red}-42\%
			&\color{Red}-42\%
			\\
			
			Best Random Transpose &\color{OliveGreen}99\%
			&\color{OliveGreen}99\%
			&\color{OliveGreen}99\%
			&\color{OliveGreen}99\%
			\\ 
			Best Random Exchange  &\color{OliveGreen}31\%
			&\color{OliveGreen}47\%
			&\color{OliveGreen}48\%
			&\color{OliveGreen}48\%
			\\
			Best Random Insert  &\color{Red}-19\%
			&\color{OliveGreen}9\%
			&\color{OliveGreen}9\%
			&\color{OliveGreen}10\%
			\\ 
			Best Srz Transpose &\color{OliveGreen}99\%
			&\color{OliveGreen}99\%
			&\color{OliveGreen}99\%
			&\color{OliveGreen}99\%
			\\ 
			Best Srz Exchange  &\color{OliveGreen}55\%
			&\color{OliveGreen}66\%
			&\color{OliveGreen}66\%
			&\color{OliveGreen}66\%
			\\ 
			Best Srz Insert  &\color{OliveGreen}24\%
			&\color{OliveGreen}42\%
			&\color{OliveGreen}43\%
			&\color{OliveGreen}43\%
			\\ 
			
			
			\bottomrule
		\end{tabular}
	\end{table}
	
	In most cases VND is faster then the algorithms with a single neighborhood, however first random insert/exchange are the dominant algorithms.
	The VND algorithms are faster then the best algorithms using a single neighborhood, there is only a big difference in relative score with the first random insert algorithm. 
	
	\section{Statistical tests exercise 1}
	
	
	
	
	
	When conducting a T-test the following assumptions should be made :
	
	\begin{enumerate}  
		\item \textbf{Normal Distribution}\newline
		The data in both of the samples is normally distributed.
		We can test the distribution by using a Shapiro hypothesis test with a confidence interval of 0.05.
		
		
		\item \textbf{Similar Variance}\newline
		In most cases it is possible to calculate the variance for both populations and determine if their is a difference.
		Equal variances can also be tested by using a F-test.
		
		
		\item \textbf{Equal Data}\newline
		The size of both distributions should be the same.
	\end{enumerate}
	
	
	When the assumption of equal variance cannot be made, then the alternative is the Wilconson test. 
	
	
	\subsection{Normal Distribution}
	We will test the normal distribution assumption in both datasets by using a Shapiro Normality test.
	
	\subsubsection{Shapiro test relative scores dataset}
	\begin{table}[H]
		\centering
		\caption{Shapiro normality test relative scores with confidence of 0.05}
		\label{tab:table1}
		\begin{tabular}{l{|}lll}
			\toprule
			Algorithm & Random & Srz\\
			\midrule
			First Transpose &\color{Red}0.4116&\color{Red}0.1331\\ 
			First Exchange  &\color{Red}0.454&\color{Red}0.0986\\ 
			First Insert 	&\color{Red}0.7774&\color{Red}0.07031\\ 
			Best Transpose &\color{Red}0.3622&\color{Red}0.3252\\ 
			Best Exchange  &\color{Red}0.3106&\color{Red}0.729\\ 
			Best Insert 	&\color{Red}0.3021&\color{Red}0.4731\\	
			\bottomrule
		\end{tabular}
	\end{table}
	
	The green colored values in the tables indicate that the P-value is greater then 0.05 and thus we accept the null hypothesis.
	We conclude that the relative scores are normally distributed.
	
	\subsubsection{Shapiro test computation time dataset}
	\begin{table}[H]
		\centering
		\caption{Shapiro normality test computation time with confidence of 0.05}
		\label{tab:table1}
		\begin{tabular}{l{|}lll}
			\toprule
			Algorithm & Random & Srz\\
			\midrule
			First Transpose &\color{OliveGreen}6.078e-08&\color{OliveGreen}5.695e-07\\ 
			First Exchange  &\color{OliveGreen}6.078e-08&\color{OliveGreen}4.353e-08
			\\ 
			First Insert 	&\color{OliveGreen}2.912e-09&\color{OliveGreen}2.886e-08\\ 
			Best Transpose & \color{OliveGreen}8.503e-10&\color{OliveGreen}4.661e-07\\ 
			Best Exchange  &\color{OliveGreen}8.323e-08&\color{OliveGreen} 1.223e-07\\ 
			Best Insert 	&\color{OliveGreen}7.245e-10&\color{OliveGreen}1.032e-07\\	
			\bottomrule
		\end{tabular}
	\end{table}
	
	The computation dataset is not normally distributed, However
	it is still possible to get decent results.\newline
	When the sample size is large enough, the central limit theorem proves that the distribution of the mean of data from any distribution approaches the normal distribution.\newline
	We will be using non-parametric tests for this dataset, because these tests work with different distributions.
	
	
	
	
	
	\subsection{Which initial solution is preferable?}
	We will split the relative score and computation time dataset into 2 separate populations based on the random and Srz initial solution.
	\subsubsection{Equal variance relative score}
	By using a Z-test, we can determine if the assumptions of equal variance is made.
	The Z-test assumes that the variance of 2 populations is the same when the ratio between these values is roughly equal to 1.
	
	\begin{hypothesis}true ratio of variances is not equal to 1(same variance)\end{hypothesis}
	\begin{nullhypothesis}true ratio of variances is not equal to 1(not the same variance)\end{nullhypothesis}
	
	A higher P-value when performing a Z-test, indicates that the null hypothesis is true.
	
	\begin{table}[H]
		\centering
		\caption{Z-test difference in variance test}
		\label{tab:table1}
		\begin{tabular}{l{|}lll}
			\toprule
			Algorithm & P-value\\
			\midrule
			First Transpose & \color{OliveGreen} 9.758e-07\\ 
			First Exchange  &\color{OliveGreen}0.007478\\ 
			First Insert 	&\color{OliveGreen}4.084e-05\\ 
			Best Transpose &\color{OliveGreen}2.568e-08\\ 
			Best Exchange  &\color{OliveGreen}0.001084\\ 
			Best Insert 	&\color{OliveGreen}0.001472\\	
			\bottomrule
		\end{tabular}
	\end{table}
	
	All of the algorithms reject the null hypothesis, we conclude that the variance is not the same.
	
	
	\subsubsection{Wilcox paired T-test}
	Because the variance is not the same, we will use a Wilcox paired T-test.
	\begin{hypothesis}There is no difference between the algorithms with random and srz initial solution.\end{hypothesis}
	\begin{nullhypothesis}There is a statistical difference between the algorithms with random and Srz initial solution.\end{nullhypothesis}
	
	\begin{table}[H]
		\centering
		\caption{Wilcox Paired T-test }
		\label{tab:table1}
		\begin{tabular}{ll{|}ll}
			\toprule
			Algorithm A& Algorithm B & Wilcox P-value\\
			\midrule
			First Transpose Random &First Transpose Srz&\color{OliveGreen}1.671329e-11\\ 
			First Exchange Random  &First Exchange Srz&\color{OliveGreen}1.982484e-07\\ 
			First Insert Random 	&First Insert Srz&\color{OliveGreen}2.072226e-08\\  
			Best Transpose Random &Best Transpose Srz&\color{OliveGreen}1.671329e-11\\  
			Best Exchange Random  &Best Exchange Srz&\color{OliveGreen}0.001608026\\ 
			Best Insert Random 	&Best Insert Srz&\color{OliveGreen}5.907791e-09\\ 	
			\bottomrule
		\end{tabular}
	\end{table}
	
	Based on the P-values we can conclude that their is a statistical difference between both initial solutions.
	
	
	
	
	
	
	
	\subsection{Which pivoting rule generates better quality solutions and which is faster?}
	
	We will split the relative score and computation time dataset into 2 separate populations, based on the best and first pivoting rule.
	\newline The populations will be tested by using the following hypothesis :
	\begin{hypothesis}There is no difference between the algorithms with first and best pivoting rules.\end{hypothesis}
	\begin{nullhypothesis}There is a statistical difference between the algorithms with first and best pivoting rules.\end{nullhypothesis}
	
	
	
	\subsubsection{Equal variance test relative scores}
	
	Before determining the statistical test, we will test if the variance between both populations is the same by using a Z-test.
	
	\begin{table}[H]
		\centering
		\caption{Z-test difference in variance test}
		\label{tab:table1}
		\begin{tabular}{l{|}lll}
			\toprule
			Algorithm & P-value\\
			\midrule
			Transpose Random & \color{Red}0.3341\\ 
			Transpose Srz  &\color{Red}0.8296\\ 
			Exchange Random 	&\color{OliveGreen}0.002188\\ 
			Exchange Srz &\color{OliveGreen}0.0002616\\ 
			Insert Random  &\color{OliveGreen}9.392e-06\\ 
			Insert Srz 	&\color{OliveGreen}0.0004352\\	
			\bottomrule
		\end{tabular}
	\end{table}
	\subsubsection{Hypothesis test relative scores}
	We will use the Wilcox test on the algorithms with a different variance and the T-test on the other algorithms.
	
	
	\begin{table}[H]
		\centering
		\caption{Hypothesis testing with confidence of 0.05}
		\label{tab:table1}
		\begin{tabular}{ll{|}ll}
			\toprule
			Algorithm A & Algorithm B & Wilcox P-value & T-test P-value\\
			\midrule
			First Transpose Random &Best Transpose Random&&\color{OliveGreen}4.847919e-07\\ 
			First Transpose Srz&Best Transpose Srz&&\color{Red}0.8798656\\
			First Exchange Random&Best Exchange Random&\color{OliveGreen}1.671329e-11&\\
			First Exchange Srz&Best Exchange Srz&\color{OliveGreen}9.057981e-11&\\
			First Insert Random&Best Insert Random&\color{OliveGreen}2.045479e-11&\\ 
			First Insert Srz&First Insert Srz&\color{OliveGreen}5.836319e-10&\\
			\bottomrule
		\end{tabular}
	\end{table}
	
	Most of the algorithms in the hypothesis testing have a small P-values thus we reject the null hypothesis.
	We conclude that there is a significant difference between the First and Best pivoting rule.
	
	
	\subsubsection{Hypothesis test computation time}
	Because our data is not normally distributed, we will use a wilcox test.
	
	
	\begin{table}[H]
		\centering
		\caption{Hypothesis testing with confidence of 0.05}
		\label{tab:table1}
		\begin{tabular}{ll{|}l}
			\toprule
			Algorithm A & Algorithm B & Wilcox P-value\\
			\midrule
			First Transpose Random &Best Transpose Random&\color{OliveGreen}2.354955e-08\\ 
			First Transpose Srz&Best Transpose Srz&\color{OliveGreen}8.865947e-07\\
			First Exchange Random&Best Exchange Random&\color{OliveGreen}1.671067e-11\\
			First Exchange Srz&Best Exchange Srz&\color{OliveGreen}1.670019e-11\\
			First Insert Random&Best Insert Random&\color{OliveGreen}3.394998e-10\\ 
			First Insert Srz&Best Insert Srz&\color{OliveGreen}2.261459e-11\\
			\bottomrule
		\end{tabular}
	\end{table}
	The null hypothesis is rejected, the computation time between first and best insert is not the same.
	These results are obviously correct, because the best improvement algorithms have to search the complete neighborhood space.
	
	
	
	\subsection{Which neighborhood generates better quality solution and what computation time is required to reach local optima?}
	We will split the relative scores dataset into 3 populations, based on the different neighborhoods.
	Because we are working with multiple populations, we will be working with different hypothesis tests.\newline
	
	
	\subsubsection{Equal variance test relative scores}
	When comparing the variance between 3 populations, we will be using the Bartlett test which is an alternative to the 2 samples Z-test.
	\begin{table}[H]
		\centering
		\caption{Bartlett test with confidence of 0.05}
		\label{tab:table1}
		\begin{tabular}{lll{|}ll}
			\toprule
			Algorithm A & Algorithm B & Algorithm C & P-value\\
			\midrule
			First Transpose Random&First Exchange Random& First Insert Random&\color{OliveGreen}2.2e-16\\ 
			First Transpose Srz&First Exchange Srz& First Insert Srz&\color{OliveGreen}2.2e-16\\
			Best Transpose Random&Best Exchange Random& Best Insert Random&\color{OliveGreen}2.2e-16\\
			Best Transpose Srz&Best Exchange Srz& Best Insert Srz&\color{OliveGreen}6.496e-16\\
			
			\bottomrule
		\end{tabular}
	\end{table}
	
	We can conclude that the null hypothesis gets rejected, the variance is not the same.
	\subsubsection{Kruskall Wallis Hypothesis test relative scores}
	The Kruskall Wallis test is the alternative of the Wilcox Test when working with multiple populations.
	
	\begin{table}[H]
		\centering
		\caption{Kruskall Wallis test with confidence of 0.05}
		\label{tab:table1}
		\begin{tabular}{llll}
			\toprule
			Algorithm A & Algorithm B & Algorithm C & P-value\\
			\midrule
			First Transpose Random&First Exchange Random& First Insert Random&\color{OliveGreen}2.2e-16\\ 
			First Transpose Srz&First Exchange Srz& First Insert Srz&\color{OliveGreen}2.2e-16\\
			Best Transpose Random&Best Exchange Random& Best Insert Random&\color{OliveGreen}2.2e-16\\
			Best Transpose Srz&Best Exchange Srz& Best Insert Srz&\color{OliveGreen}2.2e-16\\
			
			\bottomrule
		\end{tabular}
	\end{table}
	
	We reject the null hypothesis and conclude that there is a significant difference between the neighborhoods.
	
	\subsubsection{Kruskall Wallis Hypothesis test computation time}
	
	\begin{table}[H]
		\centering
		\caption{Kruskall Wallis test with confidence of 0.05}
		\label{tab:table1}
		\begin{tabular}{llll}
			\toprule
			Algorithm A & Algorithm B & Algorithm C & P-value\\
			\midrule
			First Transpose Random&First Exchange Random& First Insert Random&\color{OliveGreen}2.2e-16\\ 
			First Transpose Srz&First Exchange Srz& First Insert Srz&\color{OliveGreen}2.2e-16\\
			Best Transpose Random&Best Exchange Random& Best Insert Random&\color{OliveGreen}2.2e-16\\
			Best Transpose Srz&Best Exchange Srz& Best Insert Srz&\color{OliveGreen}2.2e-16\\
			
			\bottomrule
		\end{tabular}
	\end{table}
	
	We reject the null hypothesis and conclude that there is a significant difference between the computation times of the algorithms with a certain neighborhood.
	
	
	\section{Statistical tests exercise 2}
	The statistical tests performed in on the variable neighborhood descent algorithms are the same as in part 1.\newline
	The order of the neighborhood is as follows : \newline
	Vnd1 = Transpose Exchange Insert \newline
	Vnd2 = Transpose Insert Exchange 
	\newline
	\subsection{Normal Distribution}
	We will test the normal distribution assumption in both datasets by using a Shapiro Normality test.
	
	\subsubsection{Shapiro test relative scores dataset}
	\begin{table}[H]
		\centering
		\caption{Shapiro normality test relative scores with confidence of 0.05}
		\label{tab:table1}
		\begin{tabular}{l{|}lll}
			\toprule
			Algorithm & Random & Srz\\
			\midrule
			Transpose Exchange Insert &\color{Red}0.2475&\color{Red}0.2366\\ 
			Transpose Insert Exchange  &\color{Red}0.1103&\color{Red}0.1638\\ 
			\bottomrule
		\end{tabular}
	\end{table}
	
	
	We conclude that the relative scores are normally distributed.
	
	\subsubsection{Shapiro test computation time dataset}
	\begin{table}[H]
		\centering
		\caption{Shapiro normality test relative scores with confidence of 0.05}
		\label{tab:table1}
		\begin{tabular}{l{|}lll}
			\toprule
			Algorithm & Random & Srz\\
			\midrule
			Transpose Exchange Insert &\color{OliveGreen} 8.22e-09&\color{OliveGreen}8.435e-07\\ 
			Transpose Insert Exchange  &\color{OliveGreen}2.832e-09&\color{OliveGreen}1.593e-07\\ 
			\bottomrule
		\end{tabular}
	\end{table}
	
	
	We conclude that the computation times are not normally distributed.
	
	\subsection{Which initial solution is preferable?}
	
	\subsubsection{Equal variance relative score}
	\begin{table}[H]
		\centering
		\caption{Z-test difference in variance test}
		\label{tab:table1}
		\begin{tabular}{l{|}lll}
			\toprule
			Algorithm & P-value\\
			\midrule
			First Vnd1 & \color{OliveGreen} 6.755e-10\\ 
			First Vnd2  &\color{Red}0.1125\\ 
			\bottomrule
		\end{tabular}
	\end{table}
	
	\subsubsection{Hypothesis test relative score}
	
	\begin{table}[H]
		\centering
		\caption{Wilcox Paired T-test }
		\label{tab:table1}
		\begin{tabular}{ll{|}ll}
			\toprule
			Algorithm A& Algorithm B & T-test P-value & Wilcox P-value\\
			\midrule
			First Vnd1 Random &First Vnd1 Srz&\color{OliveGreen}6.532386e-08&\\ 
			First Vnd2 Random  &First Vnd2 Srz&&\color{Red}0.9033216\\ 
			
			\bottomrule
		\end{tabular}
	\end{table}
	There is a significant difference between the initial solution in the Variable Neighborhood Descent version 1.
	Variable Neighborhood Descent Version 2 is not affected by the initial solution.
	
	
	\subsection{Which neighborhood generates better quality solution and what computation time is required to reach local optima?}
	
	
	\subsubsection{Equal variance test relative scores}
	
	\begin{table}[H]
		\centering
		\caption{Z-test with confidence of 0.05}
		\label{tab:table1}
		\begin{tabular}{lll}
			\toprule
			Algorithm A & Algorithm B & P-value\\
			\midrule
			First Vnd1 Random&First Vnd2 Random&\color{OliveGreen}0.0002654\\ 
			First Vnd1 Srz&First Vnd2 Srz&\color{Red}0.2308\\ 
			
			\bottomrule
		\end{tabular}
	\end{table}
	
	
	\subsubsection{Hypothesis test relative scores}
	
	
	\begin{table}[H]
		\centering
		\caption{Hypothesis test with confidence of 0.05}
		\label{tab:table1}
		\begin{tabular}{llll}
			\toprule
			Algorithm A & Algorithm B & P-value T-test & P-value Wilcox\\
			\midrule
			First Vnd1 Random&First Vnd2 Random&\color{OliveGreen}1.013337e-15&\\
			First Vnd1 Srz&First Vnd2 Srz&&\color{Red}0.771217\\
			
			\bottomrule
		\end{tabular}
	\end{table}
	
	There is a significant difference between the order of the neighborhoods when performing iterative improvement with a random start.
	Between the relative scores of the neighborhood order with the Rsz heuristic, there is not a significant difference.
	
	\subsubsection{Hypothesis test computation time}
	
	\begin{table}[H]
		\centering
		\caption{Hypothesis test with confidence of 0.05}
		\label{tab:table1}
		\begin{tabular}{ll{|}l}
			\toprule
			Algorithm A & Algorithm B & P-value\\
			\midrule
			First Vnd1 Random&First Vnd2 Random&\color{OliveGreen}4.916098e-09\\ 
			First Vnd1 Srz&First Vnd2 Srz&\color{OliveGreen}0.0003050896\\ 
			
			
			\bottomrule
		\end{tabular}
	\end{table}
	
	We reject the null hypothesis and conclude that there is a significant difference between the computation times of the algorithms with a certain neighborhood.
	
	
	
	\section{Conclusion}


	We conclude that the algorithms with a more complicated neighborhood such as insert/exchange, have a better relative score compared to the best solutions, but take more time to compute.
	The algorithms using the best improvement as a pivoting rule, are the slowest.
	
	The algorithms using variable neighborhood descent with a neighborhood order of Transpose Exchange Insert is faster in general but also performs in a decent amount of time.\newline
	It is obvious that performing multiple first improvements of different neighborhoods, is better then using best improvements with expensive computation time, these methods have a fast computation and have a decent relative score.
	
\end{document}